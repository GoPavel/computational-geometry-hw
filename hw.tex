\documentclass[12pt,a4paper,oneside]{article}

\usepackage[utf8]{inputenc}
\usepackage[english,russian]{babel}
\usepackage{hyperref}
\usepackage{datetime}
\usepackage{amsmath}
\usepackage{amssymb}


\begin{document}

\begin{flushright}

	{\large собрано {\today} в {\currenttime}}

\end{flushright}

\begin{center}
	{\Large \bf Домашние задания по вычислительной геометрии.}
\end{center}

\section*{Лекция 4}

	\subsection*{Попробовать реализовать:}
	\begin{itemize}
		\item Алгоритм Грэхема для построения выпуклой оболочки \textbf{in-place}.
		\item Онлайн алгоритм для нахождения выпуклой оболочки с использованием декартовых деревьев.
	\end{itemize}
	\subsection*{Доказать:}
	\begin{itemize}
		\item $l_a\cap l_b = c$, тогда если (') - двойственное отображение, то $c'$ - прямая проходящая через точки $l_a'$ и $l_b'$. Желательно доказать алгебраическим способом.
		\item Двойственную биекцию между пересечением полуплоскостей и построением выпуклой оболочки.
	\end{itemize}



\end{document}
