\documentclass[12pt,a4paper,oneside]{article}

\usepackage[utf8]{inputenc}
\usepackage[english,russian]{babel}
\usepackage{hyperref}
\usepackage{datetime}
\usepackage{amsmath}
\usepackage{amssymb}


\begin{document}

\begin{flushright}

	{\large собрано {\today} в {\currenttime}}

\end{flushright}

\begin{center}
	{\Large \bf Домашние задания по вычислительной геометрии.}
\end{center}

Задания под (*) - задания для себя
\section*{Лекция 1}
	Без практики

\section*{Лекция 2}
	Без практики

\section*{Лекция 3}
	\begin{enumerate}
		\item Доказать что произведение афинных пространств - афинное пространство.
		\item $l_a\cap l_b = c$, тогда если (') - двойственное отображение, то $c'$ - прямая проходящая через точки $l_a'$ и $l_b'$.
		\item Доказать, что пересечение плоскостей может быть получено с помощью построения выпуклой оболочки в двойственной геометрии.
		\item (*) Рассмотреть принадлежность эллипсу и сферам высших порядков таким же способом, как для круга.
	\end{enumerate}
\section*{Лекция 4}
	\begin{enumerate}
		\item Доказать корректность алгоритма поиска касательных к двум выпуклым оболочкам за $O(n)$.
		\item Придумать алгоритм поиска касательной к выпуклой оболочке через тернарный поиск.
		\item Вспомним задачу достроения многоугольника до выпуклого. На лекции было дано утверждения для доказательства алгоритма: \textit{на $i$-ом шаге $\exists A \subset S_i$ - подмножество точек, которые образуют начало выпуклой оболочки, а остальные точки уже точно не будут находиться в ней.} \\
		Докажите данное утверждение используя факты о том, что мы начинаем на точке из оболочки, и то, что линия без самопересечений.
		\item (*) Реализовать Грэхема in-place и динамическую выпуклую оболочку.
	\end{enumerate}
\section*{Лекция 5}
	\begin{enumerate}
		\item Обобщить $mergehull$ и $Chen's$ в 3d.
		\item Рассмотреть случай когда пересечение полуплоскостей будет бесконечным.
		\item (*) Написать выпуклую оболочку для 4d.
	\end{enumerate}

\end{document}
